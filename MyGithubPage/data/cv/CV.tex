\documentclass{crcv}
\usepackage{amstext,amsmath}
\usepackage{color}
\usepackage{graphicx}
\usepackage{longtable}
\usepackage{rotating}
\usepackage{fancyhdr}
\usepackage{amssymb}
\usepackage{picins}
\usepackage{epsfig}
\usepackage{hyperref}
\usepackage{fourier-orns}

\definecolor{darkblue}{rgb}{0.0,0.0,0.3}
\definecolor{fortitle}{rgb}{0.0,0.0,0.5}
\hypersetup{colorlinks,breaklinks,linkcolor=darkblue,urlcolor=darkblue,anchorcolor=darkblue,citecolor=darkblue}

\newcommand{\desc}[4]{
\begin{description}
\item[\textcolor{fortitle}{#1}]\hspace*{\fill}\textit{\textcolor{fortitle}{#2 -- #3: #4}}\\
\end{description}
}

\newcommand{\descC}[3]{
\begin{description}
\item[\textcolor{fortitle}{#1}]\hspace*{\fill}\textit{\textcolor{fortitle}{#2: #3}}\\
\end{description}
}

\newcommand{\descB}[2]{
\begin{description}
\item[\textcolor{fortitle}{#1}]\hspace*{\fill}\textit{\textcolor{fortitle}{#2}}\\
\end{description}
}

\begin{document}
\begin{center}{\Huge \textsc{Quentin \textcolor{fortitle}{Quadrat}} \large Computer Engineer in Real Time and Embedded Systems}\vspace{2mm}\end{center}

\MakeEntete

%\begin{center}Objective: challenging job in real time computing for embedded systems.\end{center}

%% =========================================================================================================
\section{\textcolor{fortitle}{Sum}mary}

\begin{itemize}
\item[$\bullet$] Come from EPITA a computer science engineering school with a mainly
  C/Unix oriented background.
\item[$\bullet$] Specialization in programming microcontrollers.
\item[$\bullet$] Good skills on motor control, basic skills on Inertial Measurement Unit.
\item[$\bullet$] Explored by myself other domains like electronics or
  feedback and automatic control systems (Scilab/Scicos).
\item[$\bullet$] Worked on several drones successfully commercialized.
\end{itemize}

%% =========================================================================================================
\section{\textcolor{fortitle}{Soft}ware Development Experience}

%% ---------------------------------------------------------------------------------------------------------
\desc{Parrot Faurecia Automotive}{2017}{today}{Automotive middleware}

\textit{Development for a proof of concept initially for
Renault-Nissan then transformed as real project for McLaren.}

\begin{itemize}
\item[$\bullet$] C/C++ development in hypervised solution with 3
  virtual machines: Linux, Android and real time OS.
\item[$\bullet$] C++/java development in the Android (N.car then O) car
  services framework. Implemented the Vehicle Hardware Abstraction Layer
  (VHAL) aka an abstraction of the physical transport layer (like
  CAN/LIN) for other Android services (java).
\item[$\bullet$] Qt and QML development of a proof of concept car
  cluster/dashboard displaying vehicle information from the VHAL
  and tested on a concrete car.
\item[$\bullet$] Linked the vehicle climate actuator to the Android
  climate control application through the VHAL.
\item[$\bullet$] Esterel, C development of the prototype of an ASIL
  real time application supervising the cluster display through the
  support of hardware.
\item[$\bullet$] Made protocol libraries for allowing virtual machines
  to communicate together (display output checker).
\item[$\bullet$] Received a formation on ISO-26262.
\item[$\bullet$] Wrote a methodology for knowing if a proposed
  architecture will run in real-time (Petri, Grafcet, Max-Plus
  algebra).
\item[$\bullet$] Made the portage of Android N.car car services to
  Android L for an older Parrot project.
\item[$\bullet$] C, C++, Java, QML, Qt, Ethernet socket, unit
  tests. Architecture specification. Methodology Scrum, Gerrit, Git,
  Docker, gdb server.
\end{itemize}

%% ---------------------------------------------------------------------------------------------------------
%\hrulefill\hspace{0.2cm} \floweroneleft\floweroneright \hspace{0.2cm} \hrulefill
\desc{Parrot Drones}{Jan. 2011}{Jan. 2016}{Microcontroller developer}

\textit{Development alone on the firmware of several generations of
  Electronic Speed Controller (ESC) for the following drones: AR Drone
  1 and 2, Bebop 1 and 2, Exom 1, Disco 1, Anafi. ESC are boards
  driving brushless motor with a propeller.}

\begin{itemize}
\item[$\bullet$] Wrote the specification document (SA/RT).
\item[$\bullet$] Wrote the brushless motor model and its regulator
  in Scicos.
\item[$\bullet$] Developed multiple firmware based on the 6 steps
  motor commutation. Motors speed are drive with a closed loop
  regulation for speed: PID, feedforward, Kalman filter for the speed
  estimation. Power On Self Test are made like detecting if the motor
  is plugged to its ESC. C/ASM for Atmel Atmega8. C and logic gates
  programming (FPGA-style) for Cypress PoSC3 and Cypress PoSC5LP. A
  single chip drives four motors.
\item[$\bullet$] Developed bootloader firmware, bootloader protocol,
  and flasher tool. Motor firmware are upgraded when the drone user
  updates its smart-phone application.
\item[$\bullet$] Developed protocols based on I$^2$C, spi, UART
  (C/Linux/$\mu$C).
\item[$\bullet$] Developed motor test bench controlling peripherals
  (power supply, tachymeters...).
\item[$\bullet$] Worked with the hardware team for evolving the
  hardware of the ESC.
\end{itemize}

%% ---------------------------------------------------------------------------------------------------------
%\hrulefill\hspace{0.2cm} \floweroneleft\floweroneright \hspace{0.2cm} \hrulefill
\desc{Parrot Drones}{Jan. 2016}{Jan. 2017}{Microcontroller developer}

\textit{Evolution of motor commutation. Worked on Field Oriented
  Control (FOC) solution: motor torque and speed are estimated and
  controlled.}

\begin{itemize}
\item[$\bullet$] Internship supervisor for implementing Field Oriented
  Control firmware for controlling a motor in angle for a gimbal
  project.
\item[$\bullet$] Wrote a FOC model with Scicos for simulation.
\item[$\bullet$] Firmware finally not developed internally by Parrot.
\item[$\bullet$] Supervized the firmware delivery (developed by Active
  Semi PAC2553). In contact with a US engineer for the firmware
  evolution and development.
\item[$\bullet$] Integrated the firmware for controlling in position a
  gyro-stabilized pod (aka a camera gimbal). Integrated the firmware
  for spinning propellers.
\item[$\bullet$] Developed bootloader firmware, bootloader protocol,
  bootloader tool and ICSP flasher tool based on STM board.
\end{itemize}

%% ---------------------------------------------------------------------------------------------------------
%\hrulefill\hspace{0.2cm} \floweroneleft\floweroneright \hspace{0.2cm} \hrulefill
\descC{Parrot Drones}{2 months}{Thermal calibration for IMU sensors}

\textit{Development of a fast IMU thermal calibration process embedded
  in the drone. Project constraint: thermal calibration is made during
  the factory process and shall be as fast as possible.}

\begin{itemize}
\item[$\bullet$] Studied the state of art, worked with thermal chamber
  and turntable pods, write specification document for the factory
  process.
\item[$\bullet$] Sensors thermal biases are removed.
\item[$\bullet$] The IMU temperature is maintained fixed by the drone
  (closed loop regulation) during its fly.
\item[$\bullet$] Worked with hardware team, developed driver for
  reading sensors.
\item[$\bullet$] \href{https://bases-brevets.inpi.fr/en/document-en/FR3037672.html}{One patent realized}.
\end{itemize}

%% ---------------------------------------------------------------------------------------------------------
%\hrulefill\hspace{0.2cm} \floweroneleft\floweroneright \hspace{0.2cm} \hrulefill
\descB{Parrot Drones}{Canceled drone based sub-projects}
\begin{itemize}
\item[$\bullet$] Read outputs of different remote controls like PPM,
  PPM-Sumed, Spektrum by an ATmega8 (canceled project).
\item[$\bullet$] Reading the state of art for Simultaneous
  Localization And Mapping (SLAM) algorithm.
\end{itemize}

%% ---------------------------------------------------------------------------------------------------------
%\hrulefill\hspace{0.2cm} \floweroneleft\floweroneright \hspace{0.2cm} \hrulefill
\desc{Eurogiciel}{2007}{2010}{Consulting for Sagem}

\textit{Worked on fixing bugs in the code source of the Airbus A400M
  inertial measurement unit (IMU) made by Sagem.}

\begin{itemize}
\item[$\bullet$] Analysis the IMU code source, locate and fix the bug.
\item[$\bullet$] Reproduce the issue on a test bench by writing
  non-regression scripts, summarized with document..
\item[$\bullet$] Respect the DO-178B norm.
\item[$\bullet$] Got a formation on the Esterel V6 software.
\end{itemize}

%% =========================================================================================================
\section{\textcolor{fortitle}{Int}ernships}

%% ---------------------------------------------------------------------------------------------------------
\desc{INRIA}{September}{December 2005}{Code generation}

\textit{Four months internship at the National Institute for Research
  in Computer Science and Automatic Control on the development of
  \href{http://www.syndex.org/}{SynDEx} a software used for generating
  the code of distributed real time applications.}

\begin{itemize}
\item[$\bullet$] Development in OCaml, CamlTk, OCamllex, OCamlYacc and
  M4 for evolving GUI, lexer and parser for allowing it to generate
  the hosted code source that the user can store inside the GUI.
\item[$\bullet$] Learning the law control and writing tutorials for
  SynDEx where cars are following the previous car.
\end{itemize}

%% ---------------------------------------------------------------------------------------------------------
%\hrulefill\hspace{0.2cm} \floweroneleft\floweroneright \hspace{0.2cm} \hrulefill
\desc{INRIA}{January}{July 2007}{Car platooning}

\textit{Continuation of the previous internship. Development of an
  algorithm, based on the video stream of a single low cost camera,
  for platooning electric cars named CyCab.}

\begin{itemize}
\item[$\bullet$] Retro-engineering the CyCab electronic architecture
  (RTAI Linux and MPC555 boards) as well as its law control (C and
  assembly code).
\item[$\bullet$] Implement the original CyCab law control in Scicos
  and translated it into SynDEx.
\item[$\bullet$] Implement the platooning algorithm based on image
  processing (PID control for the maintaining the distance with the
  previous CyCab, Kalman filter for tracking the CyCab in the
  picture), read pictures from a firewire camera.
\item[$\bullet$] Buy a newer embedded desktop allowing a RTAI Linux
  and for displaying the camera.
\item[$\bullet$] Test on real situation with the CyCab.
\end{itemize}

%% =========================================================================================================
\section{\textcolor{fortitle}{Edu}cation}
\begin{description}
\item[\textcolor{fortitle}{2001--2007}] EPITA school: Ecole Pour l'Informatique et les Techniques Avanc\'ees.
% C/Unix oriented background with a specialization in real time computing and embedded systems.
%\item[2001] High school diploma in science (Baccalaur\'eat S).
\end{description}

\begin{description}
\item[\textcolor{fortitle}{French}] Mother tongue.
\item[\textcolor{fortitle}{English}] Fluent, 775 at TOEIC.
\item[\textcolor{fortitle}{Spanish}] Beginner.
\end{description}

%% =========================================================================================================
\section{\textcolor{fortitle}{Comp}uter skills}
\begin{description}
\item[\textcolor{fortitle}{Languages}] C, C++11, Esterel, OCaml, Delphi, Forth, currently learning Julia.
\item[\textcolor{fortitle}{GNU/Linux}] git, bash, M4, Makefile, gdb(server), valgrind, emacs, flex/bison.
\item[\textcolor{fortitle}{Lib}] GTKmm, OpenGL core, \LaTeX.
\item[\textcolor{fortitle}{Tests}] gmock/gtest, crpcut, cppunit, gcov.
\item[\textcolor{fortitle}{CI}] Travis-CI, OpenSuse Build, coverity scan.
\item[\textcolor{fortitle}{Sys.\;Analysis}] Real-Time Structured methods (SART), plantuml.
\item[\textcolor{fortitle}{Embedded}] Digital electronics, oscilloscopes, Cypress PsoC3/5,
  Active Semi PAC2553, Cortex M0, PIC16F (assembly language), dsPIC30F
  (C language), UART, I$^2$C, SPI.
\item[\textcolor{fortitle}{Control}] Scilab, Scicos, SynDEx, Automatic control systems learnt with internships and projects.
\item[\textcolor{fortitle}{Norms}] ISO-26262, DO-178B.
\end{description}

%% =========================================================================================================
\section{\textcolor{fortitle}{Pers}onal Github projects}

\begin{description}
\item[\textcolor{fortitle}{Chess}] Chess project in C++ for learning
  neural network. The AI shall learn how to move pieces on the
  chessboard.  Learning Julia language and frameworks like Knet and
  TensorFlow.
\end{description}

\begin{description}
\item[\textcolor{fortitle}{SimTaDyn}] Project for creating dynamic
  geographic maps and manipulate them as spreadsheet. C++, Forth,
  GTKmm, OpenGL core.
\end{description}

%% =========================================================================================================
\section{\textcolor{fortitle}{EPITA} projects}

\begin{description}
\item[\textcolor{fortitle}{Helicopter}] Studies and realization of a
  four-rotor heads micro indoor model helicopter~: computer-based
  control (Scilab) and embedded system (2 dsPIC micro-controllers,
  acceleration sensor, 4 brushed motor drivers) 1 year, 1 people,
  annual report.
\end{description}

\begin{description}
\item[\textcolor{fortitle}{Car}] 3D simulator game made in Delphi and
  OpenGL. The player drives a car inside a city with its traffic
  jam. The car dynamic is simulated (1 year, 2 peoples, annual
  report).
\end{description}

\begin{description}
\item[\textcolor{fortitle}{Processor}] Study and simulate of a CISC
  micro-processor based on an FPGA emulation software (MaxPlus2).
\end{description}

\begin{description}
\item[\textcolor{fortitle}{Bash}] Developed an Unix Shell commands
  interpreter in C (1 month, 6 peoples).
\end{description}

\begin{description}
\item[\textcolor{fortitle}{Tiger}] Developed a Tiger language compiler
  in C++, Flex and Bison (4 months, 4 peoples).
\end{description}

\begin{description}
\item[\textcolor{fortitle}{Lisp}] Developed a Common Lisp language
  interpreter in OCaml (1 week, 3 peoples).
\end{description}

\begin{description}
\item[\textcolor{fortitle}{Forth}] Developed a Forth language
  interpreter in C, for the SimTaDyn project (alone). Forth is a stack
  language running inside a virtual machine. Forth has no syntax but
  the syntax can self evolve.
\end{description}

\begin{description}
\item[\textcolor{fortitle}{Corewar}] Developed in C a virtual machine
  executing assembly language files simulating concurrent viruses with
  the theme a race tournament.
\end{description}

\begin{description}
\item[\textcolor{fortitle}{Recalage}] Find a rigid transformation
  which matches a cloud of points with a surface (2 weeks, 1 people).
\end{description}

\begin{description}
\item[\textcolor{fortitle}{PDE}] Solver of partial derivative
  equations with the finished difference method written in C.
\end{description}

\begin{description}
\item[\textcolor{fortitle}{Bistro}] Big numbers calculator written in
  C (2 weeks, 2 peoples).
\end{description}

%% =========================================================================================================
\section{\textcolor{fortitle}{Oth}er}
Fish tank, chess player, role play games.

\end{document}
